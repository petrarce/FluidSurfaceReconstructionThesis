\documentclass[
	11pt, 
	DIV10,
	a4paper, 
	oneside, 
	headings=normal, 
	captions=tableheading,
	final, 
	numbers=noenddot
]{scrartcl}

\usepackage{lipsum}
\usepackage{graphicx}
\usepackage{amsmath}
\usepackage{subcaption}
\usepackage{listings,amsfonts}
\usepackage{pdfpages}
\usepackage[utf8]{inputenc}
\usepackage[export]{adjustbox}
\usepackage[]{algorithm2e}
\usepackage{algorithmic}
\usepackage{amssymb}

\SetAlFnt{\small}
\algsetup{linenosize=\tiny}

\title{General notes during thesis work}
\author{Iohannes Folbort}
\begin{document}

Algorythm of grid-based blurring\newline
\begin{algorithm}[H]
\KwData {OldLevelSet - given level set values, UserSmoothingFactor - smoothing factor, provided from user, FluidParticlesCount - number of particles in the neighbourhood of grid cell}
\KwResult{BlurredLevelSet}
\Begin{
	$maxNeighbourCount = \dfrac{CellSupportRadius^3}{FluidParticleRadius^3}$\newline
	\For {$vert \in GridVertices$} {

		$neighbors = findCellNeighbors(vert)$\newline
		$newLevelSetValue = OldLevelSet[vert]$\newline
		\For { $neigborVert \in neighbors$} {
			$newLevelSetValue = newLevelSetValue + OldLevelSet[neigborVert]$\newline
		}
		$newLevelSetValue = \dfrac{newLevelSetValue}{neighbors.size()} + 1$\newline
		$factor = min(1, UserSmoothingFactor \cdot \dfrac{FluidParticlesCount[vert]}{maxNeighbourCount}$\newline
		$BlurredLevelSet[vert] = OldLevelSet[vert] \cdot (1 - factor) + factor \cdot newLevelSetValue$\newline
	}
	$return BlurredLevelSet$
}
\end{algorithm}


Algorythm of grid-based Mls smoothing\newline
\begin{algorithm}[H]
\KwData {OldLevelSet - given level set values, SimVal- similarity value, MCgrid - marching cubes grid}
\KwResult{NewLevelSet}
\Begin{
	\For {$vertice \in MCgrid$} {
		verticeCurvature = curvature(vertice);\newline
		\If {$|verticeCurvature| < 0.5$}{
			kernelOffset *= 2;\newline
			kernelSize *= 2;\newline
		} \ElseIf {$|verticeCurvature| < 1.5$} {
			kernelSize *= 2;\newline
		}

		samples = getNeighbors(vertice, kernelSize, kernelOffset, SimVal)\newline
		NewLevelSet[vertice] = applyMlsCorrection(vertice, samples, kernelSize, kernelOffset)\newline

	}
}
\end{algorithm}

getNeighbors Algorythm of grid-based Mls smoothing\newline
\begin{algorithm}[H]
\KwData {Vertice - vertice around which neighborhood is found, SimVal- threshold of how far sdf value should be for neighbor vertices, KernelSize, KernelOffset}
\KwResult{NeighborVertices}
\Begin{
	baseSdf = getSDFvalue(Vertice);\newline
	\For {$i, j, k \in -kernelSize, kernelSize$} {
		neighborVertice = baseCell + Vector(i * kernelOffset, j * kernelOffset, k * kernelOffset);\newline
		sdfValue = getSDFvalue(neighborVertice);\newline
		\If {$|sdfValue - baseSdf| > SimVal$} {
			continue;\newline
		}
		$NeighborVertices.push_back(neighborVertice)$
	}
	return NeighborVertices\newline
}
\end{algorithm}



\end{document}
