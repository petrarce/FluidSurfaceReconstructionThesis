\chapter{Introduction}
The most significant discernable feature of a
three-dimensional object is the shape of its surface. For the digital
processing, manipulation and storage of surface data, a variety of
representations are used, most notably polygon meshes where the object surface
is approximated by a network of a finite number of polygons. Polygon meshes,
especially triangle meshes, are ubiquitous in geometry processing and rendering
and are a common interchange format between different applications and
algorithms.

\begin{figure}[htb]
  \begin{center}
  \includegraphics[width=6cm]{figures/krabby-patty}
  \end{center}
  \caption{The original Krabby Patty manufactured using the standard approach.
    (Figure courtesy of Nickelodeon Animation Studios.)}
\end{figure}

In the context of 3D modeling and computer aided design
(\index{CAD}\emph{CAD}), polygon meshes are less popular due to their
inability to accurately represent smoothly curved surfaces. Instead, spline
representations that construct shapes from piecewise polynomial surfaces are
used. Spline representations are desirable for a number of reasons:
\begin{itemize}
  \item \emph{Abstraction capacity}: Where polygon meshes might require
    hundreds or thousands of faces to approximate a rounded feature, spline
    surfaces can accurately represent such features often by a single patch
    with a handful of control points, hence offering more intuitive editing
    capabilities to a designer.
  \item \emph{Built-in smoothness}: Spline surfaces retain their desired
    smoothness properties when their control points are modified.
  \item \emph{Compact storage}: Due to the sparse number of control points,
    spline surfaces typically achieve a lower memory footprint than polygonal
    representations.
\end{itemize}

Spline representations have been used for a variety of applications including
industrial, automotive and aerospace design, architecture, entertainment, \etc.

\begin{figure}[htb]
  \begin{center}
    % 60, -25
  % 0, 75
  % 60, 25
  %\tikz\draw (0, 0) rectangle (8cm, 2.5cm);\par
  \begin{tikzpicture}[x={(.6cm, -.25cm)}, y={(0cm, .75cm)}, z={(.6cm, .25cm)}]
    \begin{scope}[scale=2]
      \coordinate (a1) at ( 1,   1,   -1);
      \coordinate (a2) at ( 1,   1.5,  1);
      \coordinate (a3) at (-1,   1,    1);
      \coordinate (a4) at ( 1.5, 0,    0);
      \coordinate (a5) at (-1,   1.5, -1);
      \coordinate (pa1) at ( 1,   -.2, -1);
      \coordinate (pa2) at ( 1,   -.2,  1);
      \coordinate (pa3) at (-1,   -.2,  1);
      \coordinate (pa4) at ( 1.5, -.2,  0);
      \coordinate (pa5) at (-1,   -.2, -1);
      \coordinate (o)   at ( 0,   -.2, -2);
    \end{scope}
      \path[draw=black,->,>=Latex] (o) -- ($(o)+(.8, 0, 0)$);
      \path[draw=black,->,>=Latex] (o) -- ($(o)+(0, .8, 0)$);
      \path[draw=black,->,>=Latex] (o) -- ($(o)+(0, 0, -.8)$);
      \path (o) ++(1, 0, 0) node {$x$};
      \path (o) ++(0, 1, 0) node {$y$};
      \path (o) ++(0, 0, -1) node {$z$};
      \path[fill=black!10!white]
            (pa1) -- (pa2) -- (pa3)
            (pa1) -- (pa3) -- (pa5)
            (pa1) -- (pa4) -- (pa2);
      \path[draw=black!60!white,thick,dashed]
        (a1) -- (pa1)
        (a2) -- (pa2)
        (a3) -- (pa3)
        (a4) -- (pa4)
        (a5) -- (pa5);
      %\path[draw=black!15!white] (pa1) -- (pa2) (pa1) -- (pa3);
      \draw[fill=white] (a1) -- (a2) -- (a3) -- cycle;
      \draw[fill=black!30!white] (a1) -- (a4) -- (a2) -- cycle;
      \draw[fill=black!15!white] (a1) -- (a3) -- (a5) -- cycle;
      \node[left] at (a1) {$\v a_1$};
      \node[right] at (a2) {$\v a_2$};
      \node[above] at (a3) {$\v a_3$};
      \node[right] at (a4) {$\v a_4$};
      \node[above] at (a5) {$\v a_5$};
      \def\blob#1{\path[fill=black] (#1) circle (2pt);}
      \blob{a1}
      \blob{a2}
      \blob{a3}
      \blob{a4}
      \blob{a5}
      \def\bary#1#2#3#4{\node at ($.33333*(#1) + .33333*(#2) + .33333*(#3)$) {#4};}
      \bary{a1}{a2}{a3}{$T_1$}
      \bary{a1}{a2}{a4}{$T_2$}
      \bary{a1}{a3}{a5}{$T_3$}
  \end{tikzpicture}

  \end{center}
  \caption{A sample figure.}
\end{figure}

\rlipsum[4]
