\chapter{Related Work}\label{sec:related-work}
Previous research that is relevant to our work can be roughly categorized into
three fields: The mathematical representation of smooth surfaces
(\cref{sec:surface-representation}), the segmentation of surfaces into suitable
parametric domains (\cref{sec:quad-layout-generation}), and the automatic
approximation of 3D objects by a simplified surface model
(\cref{sec:spline-surface-fitting}).

\section{Surface Representation}\label{sec:surface-representation}
\index{B{\'e}zier curve}B{\'e}zier curves~\cite{Cas63,Bez67} and their bivariate
generalization, \index{tensor product Bezier surface@tensor product B{\'e}zier
surface}tensor product B{\'e}zier surfaces~\cite{Boo62}, provide a simple
parametric representation of polynomial curves and surfaces. Joining multiple
B{\'e}zier surfaces with $C^r$ continuity produces smooth \index{spline
surface}spline surfaces with a regular layout which are generalized by
\index{B-Spline surface}B-Spline surfaces~\cite{GR74}. \index{NURBS
surface}NURBS surfaces~\cite{Ver75} further broaden this definition by allowing
weighted B{\'e}zier points, thus providing a unified representation of both
piecewise polynomial surfaces and \index{conic sections}conic sections.  One
restriction of NURBS, the strictly grid-like arrangement of knots in the
parameter domain, is lifted by \index{T-Splines}T-Splines~\cite{SZB+03} which
allow partial rows and columns of knots, hence providing a more compact
representation for tensor product spline surfaces.

\rlipsum[3]

\section{Quad Layout Generation}\label{sec:quad-layout-generation}
\rlipsum[3]

\section{Spline Surface Fitting}\label{sec:spline-surface-fitting}
\rlipsum[3]
