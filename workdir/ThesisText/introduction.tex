\chapter{Introduction}
Liquids simulation is one of the most difficult tasks in realistic modeling. Their complexity ranges from overly time-consuming, high-quality standalone animations to real-time low-resolution systems. Fluid animation is used in a wide range of areas e.g. in the entertainment industry, scientific researches, engineering, etc. 

There are two known techniques for modeling fluid phenomena, which have their advantages and disadvantages. Lagrange methods represent a continuous liquid as a collection of discrete particles, and the properties of the liquid mentioned above are carried by these particles. In contrast to Lagrange methods in Eulerian methods, the simulation domain is discretized into mesh grids and the values of physical properties on grid points are determined by solving the governing equations. In addition to position, speed, and acceleration, pressure, density, and viscosity are among the most frequently used fluid attributes.\\
In recent years, Lagrangian or so-called mesh-free methods have become a competitive alternative to Eulerian (mesh-based) methods due to their inherent mass conservation, the flexibility of simulation in unbounded domains, and ease of implementation.\\
Additional to Lagrangian simulation, also known as SPH simulation techniques, there is another related field of fluid surface extraction and visualization. Different techniques help to extract the surface from the SPH particle set. These techniques will be described in further sections. However, these techniques require advanced parameters tunning to achieve good surface quality. In most cases smooth surfaces are obtained in terms of degraded performance, thus it is always a trade-off between surface extraction speed and smooth fluid surface construction.\\
Nevertheless, already existing reconstruction techniques are prone to generating surfaces that look bumpy and rough in the flat areas. In this thesis two surface smoothing methods were implemented, that based on existing surface reconstruction methods perform internal operations on a uniform grid, which is used for surface extraction, to smooth out the extracted surfaces and remove undesired bumps on the fluid surface.\\
The first method uses general-purpose blur techniques on a 3-dimensional grid to generate smooth surfaces. Although the approach seems pretty easy to incorporate into the surface extraction, additional efforts should be applied to preserve the small and sharp features of the fluid. The main advantage of the method is its efficiency and ease of implementation. However, this method is not always stable and prone to the generation of artifacts in case of inappropriate parametrization.\\
The second method which is introduced in this thesis is a smoothing filter based on MLS surface approximation. This technique is widely used in the field of surface reconstruction from point clouds, however, in this theses, another approach is described on the application of the MLS filter on the uniform SDF grid, which is used for fluid surface mesh extraction. This method is good in terms of smoothing scalability and much more stable to over-parametrization, however, the main disadvantage of this technique is its performance penalties due to the necessity to solve linear equations and run time-consuming neighborhood detection procedures.