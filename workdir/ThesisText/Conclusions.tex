\chapter{Conclusions}
In this thesis, two methods were presented for smoothing fluid surfaces - Blur smoothing filter and MLS smoothing filter. Both methods are operating on the statically defined MC grid that represents a signed scalar field, thus the filters can be applied to all reconstruction methods, that internally use signed scalar fields and marching cubes algorithms for fluid surface extraction in SPH simulation scenarios.\\
The blur filter method was proven to be a good choice for smoothing high-frequency bumps on the flat surface areas in the meantime preserving small and sharp features of the reconstructed surface. Many parameters are defined to be able to flexibly configure the filter to achieve the required surface quality and reconstruction speed. In general, the most applicable scenario in terms of the trade-off to surface quality is to apply kernel size and kernel offset to minimal possible value (1), take half of the kernel depth (0.5), and depending on required smoothing strength to configure the number of blur iterations. In this case, it is harder to run into the situation where SDF will be smoothed to some uniform value as the result of forming wholes in the reconstructed surface.
However, the method has limitations in terms of reconstruction stability, which leads to over-smoothing the scalar field destructing sharp features, or even generating artifacts on the flat surface areas.\\
Mls filter has other strengths and weaknesses. This filter is more flexible and robust to the parameters overestimation. This filter also can be configured to remove bumps from the flat surface areas preserving sharp features. However, the main disadvantage of the method is its computational complexity. The MLS filter brings a large overhead while smoothing the scalar field. The main bottleneck is computation clusters of samples which are used to compute local approximations of fluid surface. To achieve good reconstruction results it requires applying a large number of samples during MLS smoothing, while to generate smooth surfaces with blur filter it is enough to apply a small kernel varying number of smoothing iterations.\\
More efforts can be applied to optimize filters, develop GPGPU compliant implementation. Another idea is to train deep neural networks to filter the original scalar field and generate smoothed SDF based on the results of the MLS and Blur smoothing. In general, the methods are good for application in off-line rendering to generate good-looking fluid surfaces.